\chapter{Conclusion and outlook}
\label{chap6}

This thesis addresses the dynamic performance of the rib-stiffened vaulted floor under one person walking excitation. The main contribution of this thesis can be divided into two parts. The first part is the fundamental understanding of the floor's dynamic behavior. Firstly, an applicable and efficient analysis procedure for performance evaluation was built and verified. Based on this procedure, the unique features of this floor in terms of dynamics and their relevance to the vibration problems have been identified. As a next step, the qualitative and quantitative relation among geometric parameters, modal parameters and dynamic performance was found and clarified. These findings served as a good basis for the next part, the optimization. Two different approaches were attempted for optimization, one was based on the understanding of the mechanism behind, the other adopted a machine learning alike idea and treated the whole evaluation procedure as a black box through a surrogate model. Both approaches resulted in similar optimized figure and considerable improvements in dynamic performance. Some characteristics of this floor and associated conclusions may also apply to other types of high frequency floors. 

\section{Dynamic behavior of the floor}
The sufficiently optimized floor in terms of statics has the following dynamic features: high natural frequency, generally low and $t_v/t_r$ sensitive modal mass. Its dynamics can be problematic when the great vibration induced by the low modal mass cannot be compensated by the high natural frequency. 

The geometric parameters influence the modal property and response factor to different extents. Based on the ICO index, the relevance of geometric parameters to the modal mass has the following sequence: $t_v/t_r(+)>l(+)>l/d$. For the natural frequency, the order changes: $l(-)\approx l/d(-)>t_r/t_v$. A higher $t_v/t_r$ and a lower $l/d$ value can lead to a lower response factor by increasing the modal mass and raising the natural frequency, respectively. The $l/d$ ratio loses its significance with growing span. The span is not an influential factor as its contribution in modal mass and negative impact in natural frequency compensates each other to a great degree.

The influence of geometric parameters on the dynamic response is imposed through the modal parameters. No matter what kind of combination of geometric parameters, as long as they come to the same modal properties, the response will be identical. Among all the studied floors, the first mode is dominant(86\%-92\%) in the total response. Both modal mass and natural frequency play an important role in the dynamic response. The response factor attributed to the first mode can be expressed in form
\begin{equation}
\label{eqn:R1_conclusion}
	R_1=\frac{C}{m_1f_1^{1.5}}
\end{equation}
\noindent
where $C=3778622$. Note that this formula can only be applied when the fundamental frequency is higher than 16 Hz. 

\section{Optimization of the floor}
Since more than 70\% of the studied floors failed the acceptance criterion, measures to improve the performance should be taken. To increase the modal mass, mass was added in the middle of the floor in two ways: change the density and change the thickness. The former represents the effect of filling material, but does not function well due to the considerable drop in natural frequency. The latter way means that the added mass is part of the structure, this measure can effectively reduce the dynamic response. The choice of mass increase and region to be thickened depend on the actual situation. To refine the improvements, surrogate model based GA optimization was carried out in pursuit of the optimal mass distribution with constant mass. The comparison of two optimized floors showed that the first degree PCE model was probably not robust enough, but the optimization based on it has succeeded in improving the performance to a satisfactory degree while keeping the computational cost affordable. 


\section{Limitation and future work}
Due to the time restriction for the thesis, some issues have been explored but not thoroughly researched. One is the equation \ref{eqn:R1_conclusion}, which is originally put forward in section \ref{subsec:influence of modal parameters}. This formula lacks any rigorous derivation, especially the exponent of $f_1$ term. But it matches the data so well that it seems to have expressed the relation properly. It is also not clear, whether the constant $C$ is case specific, or generally applies to others floors as long as the prerequisite $f_1>16$ Hz is held. There exists also the possibility that this formula can be adjusted depending on which frequency weighting range it is located. If the relation between modal response and modal parameters can be expressed in a similar way, the response time history analysis can be redundant.

Another issue worth further exploration is the surrogate model based optimization. In this thesis, the most primitive form of surrogate model was built, linear regression with first order polynomial. If the order goes up, the complexity of the PCE model will raise exponentially, given that full polynomial bases are used. For a not strongly nonlinear model, however, most of the coefficients of higher order polynomials can be almost zero. So a scheme with sparse polynomial bases can be developed. In many cases, perhaps also for this case, the PCE surrogate model is not the best choice. Other surrogate models, for example Kriging model, or neural network model, may have fitted better. 

Since both the modal mass and natural frequency can greatly influence the dynamic response, and the natural frequency even with a higher factor of influence. Another way of improving the dynamic performance can be, instead of adding additional mass and jeopardizing the natural frequency, further optimization for higher natural frequency. This can be much less direct and more demanding than mass addition. 

In the actual engineering practice, more factors other than $l,l/d,t_v/t_r$ may come into play. For example, the floors are assumed to be pinned, another underlying assumption is that the beam underneath is stiff enough to restrict the vertical displacement. This is an assumption that hides many uncertainties. Besides, the square form of floors in plan is very unlikely to appear. The length/width ratio can be another factor as important as the three above listed parameters.