\chapter{Literature review}
\label{chap2}

\section{Natural frequency based traditional evaluation methods}
Many factors play a role in the nature of floor vibrations in office and residential buildings under pedestrian excitation, among them the footfall loads, the configuration of partitions, furnishings, geometric shapes of floor area, etc. These factors not only influence the dynamic excitation, but also the modal characteristics of the floor. Rational calculations of vibration amplitudes induced by dynamic forces become rather complicated and uncertain. Consequently, empirical and semi-empirical natural frequency based methods have been developed to deal with this situation since decades \cite{bachmann2012vibration}.

For low-frequency floors whose first natural frequency is under 10 Hz, since the annoying vibration amplitudes are mainly caused by a coincidence of the natural frequency with the frequency of footfall forces, the problem can be solved by keeping these frequencies away from each other. That is the principal idea of "high tuning method", one of the most important and widely adopted empirical methods in last decades \cite{bachmann2012vibration}. Historically, designers have used the natural frequency of the floor as the sole measure of acceptable performance \cite{european1992eurocode}. In the United Kingdom, the traditional approach used to design conventional floors for serviceability criteria has been to check the primary and secondary beams independently for a minimum natural frequency of 4.0 Hz \cite{smith2007design}. A sufficiently high natural frequency means that a floor is effectively 'tuned' out of the frequency range of the primary harmonic components of the walking activity, thus it is warded off the likeliness of resonant behavior. 

\section{Performance based approaches}
Early acceptability criteria that were solely based on natural frequency cannot represent a realistic assessment of the vibration likely to arise in normal service. It therefore follows that ensuring a design meet certain minimum frequency limits may still result in a floor that is unacceptable in service. Conversely, some floors under such design frame could be over-conservative \cite{smith2007design}. The need for a more robust and comprehensive performance based approach became clear.

Within Europe, the following guidelines published by different research institutes provide the possibility to evaluate the dynamic performance of floors according to the response level: P354 by SCI (The Steel Construction Institute) \cite{smith2007design}, EUR 21972 by TSR (Technical Steel Research of European Commission) \cite{sedlacek2006generalisation}, EUR 24084 by JRC (Joint Research Center of European Commission) \cite{feldmann2009design} and HiVoSS guideline by RFCS (Research Fund for Coal and Steel) \cite{feldmann2010human}. 

These guidelines help assess the dynamic performance by providing stipulation of the dynamic excitation input, clarification of response assessment process, suggestion for the acceptability criteria and examples of practical application. Based on these guidelines, it is possible to conduct a simplified evaluation with hand calculation for a regular floor as well as a complex assessment with FE modeling and response time history analysis for an unconventional floor.

They have achieved agreement in the primary process of performance assessment. They all respect the fact that human body's sensitivity to a given amplitude of vibration changes with the frequency of vibration, the frequency weighting function for response evaluation is therefore introduced. Moreover, it is differentiated between low frequency and high frequency floors. They nonetheless deal with the uncertainty in dynamic excitation differently, JRC and Hivoss adopt a probabilistic manner to describe the uncertainty in footfall amplitude and frequency, while SCI and TSR simplify the dynamic excitation as a series of deterministic values varying with time. Another difference lies in the choice of representative value of response. TSR uses both acceleration and velocity, whereas SCI only acceleration, JRC and Hivoss only velocity. Due to these distinctions the classification of vibration perception varies from one to another as well.

The evaluation guideline P354 by SCI is selected to be performed for the dynamic performance assessment in this study for three main reasons. Firstly, the stochastic process adopted by JRC and Hivoss is unnecessarily complicated and computationally intensive. In addition, the deterministic relation between input and output can help unveil the hidden associations. Secondly, the assumption of footfall input in TSR is believed to be too conservative (40\% higher than that of SCI), especially when combined with a series of conservative modeling assumptions. Finally, the SCI document provide users with important and useful hints for modeling and implementation, it also seems to have gained more practical applications an citations than other guidelines.


\section{State of the art in design codes and research}
Current European design codes do not give any certain natural frequency level above which the structure should be kept to avoid vibration problems, neither do they suggest any analysis approach for evaluation of the dynamic performance. The vibration of concrete structures, as a serviceability sate limit, is even not covered in Eurocode 2 (design of concrete structures) \cite{european2004eurocode2}. This may imply that for common concrete structures, vibration is not a noticeable problem. In Eurocode 3 (design of steel structures)  and Eurocode 4 (design of composite steel and concrete structures) , requirements for vibrations should be specified for each project and agreed with the client \cite{european2005eurocode3}\cite{european2004eurocode4}.

Varieties of research have investigated dynamic performances of different regular floor systems. But only few of them have addressed high frequency floors and give referential hints about how geometry and modal properties influence the dynamic performance.  

Arup has conducted a series of research and compared the relative vibration performance of different forms of construction (reinforced concrete, pre-stressed concrete and concrete/steel composite) for hospital use \cite{arup2004hospital}. it is concluded that Of the floors that have been designed to meet strength and deflection criteria only, the dynamic performance of the concrete designs is significantly better than that of the composite designs. It further points out that although the natural frequency is an important dynamic parameter, it does not necessarily follow that a floor with a higher frequency will have a lower dynamic response. In fact the reverse may be true if the frequency is increased by removing mass . This finding indicates the possibility that when a floor is optimized towards a statically stiff and strong structure with little material (usually accompanied with a high natural frequency), it may still show vibration problems. Some other dynamic parameters other than natural frequency must play a role in the meanwhile. 

SCI P354 makes a clear distinction between low frequency floors and high frequency floors. Equation (51) in SCI P354, repeated in equation \ref{eqn:SCI} gives a useful clue of how natural frequency and modal mass influence the response acceleration in the transient phase for a simplified assessment for steel floors:
\begin{equation}
    a_{w,rms} = 2\pi\mu_e\mu_r\frac{185}{Mf_0^{0.3}}\frac{Q}{700}\frac{1}{\sqrt{2}}W
\label{eqn:SCI}
\end{equation}
\noindent
where $f_0$ is the fundamental frequency of the floor, $M$ is the modal mass. The inverse proportionality between response acceleration and $Mf_0^{0.3}$ means that the dynamic response can be reduced by raising modal mass and fundamental frequency with less effect from the latter term. It is nevertheless not clear, whether this expression holds only for the given structural form and for a simplified assessment. Unfortunately, there are also no further hints about how to alter the two correlated parameters in an expected direction.

Even less research has been performed for vaulted floors of any kind. A theoretical, experimental or empirical benchmark in terms of modal characteristics or dynamic response associated with vaulted floors is hard to find. Therefore, there exists a need, not only within the BLOCK Research Group for the NEST HiLo project soon to be built, but also for general high frequency floors whose high frequency is achieved by removing statically redundant material, to study their dynamic performance and crucial parameters that plays a role in it. The purpose of this study is restricted to evaluate and improve the dynamic performance of the rib-stiffened vaulted floor, but it may also provide a useful insight into the dynamic behavior of other types of high frequency floors.
