\begin{abstract}
The rib-stiffened vaulted floor designed by the BLOCK Research Group at ETH Zurich can achieve adequate stiffness and bearing capacity with ultra-lightweight construction. For this statically well optimized floor, the dynamic behavior was still unknown. Natural frequency based traditional methods fail to assess its dynamic performance. The aim of this study is to obtain a fundamental understanding of the floor's dynamic behavior and to develop proper measures to improve its dynamic performance if it turns out to be unsatisfactory.

The analysis procedure for performance evaluation consists of the following steps: firstly, the generation of floor mesh models with different combinations of geometric parameters; secondly, the solution of dynamic response time history via modal superposition; finally, post-processing of the response based on the chosen guideline and comparison with acceptance criterion. After the performance evaluations of these floor models, qualitative and quantitative relations among the geometric parameters, modal parameters and dynamic performance were found and understood. 

It was shown that the most studied floor failed to meet the acceptance criterion. Two different approaches were taken to improve the performance of the floor. One was based on the acquired understanding of the mechanism behind the dynamic behavior from previous findings, the other adopted a machine learning alike idea and used a surrogate model for automatic optimization. Both approaches ended up with similar optimized figures and accomplished considerable improvements in dynamic performance. 

This thesis has revealed the dynamic characteristics of the rib-stiffened vaulted floor and possible ways to improve its dynamic performance. It may also have provided some insight into the dynamic behavior of general high frequency floors. 

 
\end{abstract}