%%% comparison and discussion of guidelines

% Approaches put forward by SCI (The Steel Construction Institute from UK)[ref.], TSR(Technical Steel Research of European Commission)[ref.] and JRC (Joint Research Center of European Commission)[ref.] together with Hivoss guideline share some similarities but also show certain differences. They all respect the fact that human body's sensitivity to a given amplitude of vibration changes with the frequency of vibration, the frequency weighting function for response evaluation is therefore introduced. They also differentiate the usage of the floor structure and establish corresponding acceptance criteria. They nonetheless deal with the uncertainty in dynamic excitation differently, JRC and Hivoss adopt a probabilistic manner to describe the uncertainty in footfall amplitude and frequency, while SCI and TSR simplify the dynamic excitation as a series of deterministic values varying with time. Another difference lies in the choice of representative value of response. TSR adopt both acceleration and velocity, whereas SCI only acceleration, JRC and Hivoss only velocity. Due to these distinctions the classification of vibration perception varies from one to another as well.

% The approach by SCI is selected to be performed for dynamic performance assessment. There are three major reasons for the choice. Firstly, the stochastic process that JRC and Hivoss adopt is deemed to be unnecessarily complex and computationally intensive. On the contrary, the deterministic relation between input and output can help unveil the hidden association. Secondly, the assumption of footfall input in TSR is believed to be too conservative (40\% higher than that of SCI), especially when combined with a series of conservative assumptions made in modelling part [xxx,ref. to that part]. Finally, SCI document provide users with important and useful hints for modelling and implementation. The detailed procedure based on SCI will be presented in [xxx evaluation of vibration perception] part.

%%% hints from Andrew
% Section 4.2 onwards should not be about "optimisation", but learning from Section 3.0 to make more of the floors acceptable for design. Answer "What can we advise the engineers to include in their floor design that will make their floors respond acceptably for footfall loading?" because without improvements they do not all work good enough.

% Remember research is about 1) identify problem (Section 1) 2) describe a methodology for research (Section 2) 3) present results (Section 3) 4) interpret the findings and examine how it has improved our understanding of the original problem (Section 4). Finally (Section 5) state the core findings and results, why the research was useful, and what needs to be done next.



In summary, the importance of geometric parameters in terms of influence on the response has the following sequence: $t_v/t_r>l/d>l$. The influence is indirectly imposed through the modal parameters. $t_v/t_r$ has great impact on the modal mass, $l/d$ can influence the natural frequency to a considerable extent. Modal mass has great significance to the response, if the natural frequency can be kept constant, the response factor will be inverse proportional to the modal mass. The natural frequency has an even greater influence on the response in the mathematical relation. The natural frequency seems to have less impact based on figure \ref{fig:m1,f1_R1_scatter}. This misconception stems from the narrower value range of natural frequency. The natural frequency ranges from 20 Hz to 100 Hz (5 times), while modal mass from 230 kg to 6700 kg (29 times), so the effect of modal mass is much more dramatic. Nonetheless, these ranges may also indicate the relative difficulty to change the two parameters.  